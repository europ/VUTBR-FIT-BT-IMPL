\subsection{\texorpdfstring{Brno University of Technology - Faculty of
Information Technology\\
Bachelor's Thesis - Implementation\\
2017/2018}{Brno University of Technology - Faculty of Information Technology Bachelor's Thesis - Implementation 2017/2018}}\label{brno-university-of-technology---faculty-of-information-technology-bachelors-thesis---implementation-20172018}

\subsubsection{Requirements}\label{requirements}

\textbf{Bot is working under Linux only!}

\textbf{Bot is customized to MY account credentials!}

To change it on yours, please see the
\href{https://github.com/europ/VUTBR-FIT-BT-IMPL\#customization}{\emph{customization}}
section.

\paragraph{Credentials}\label{credentials}

\begin{itemize}
\tightlist
\item
  \href{https://github.com/settings/tokens}{GitHub personal access
  token}

  \begin{itemize}
  \tightlist
  \item
    full access if possible to avoid unsolicited access problems
  \end{itemize}
\item
  Github organization

  \begin{itemize}
  \tightlist
  \item
    to be a member
  \end{itemize}
\item
  Organization's repository

  \begin{itemize}
  \tightlist
  \item
    the bot is set to check this repository (pull requests / builds /
    etc.)
  \end{itemize}
\item
  Your own forked repository from the organization's repository

  \begin{itemize}
  \tightlist
  \item
    to create a pull request which will be inspected by the bot (bot
    reacts to the PR's comments)
  \end{itemize}
\item
  \href{https://developer.gitter.im/apps}{Gitter personal access token}

  \begin{itemize}
  \tightlist
  \item
    do not forget to sign in
  \end{itemize}
\item
  Gitter room
\end{itemize}

\paragraph{Customization}\label{customization}

Rewrite my informations by yours in:

\begin{itemize}
\tightlist
\item
  initializing script \texttt{miqbot\_init.sh}

  \begin{itemize}
  \tightlist
  \item
    occurrence of \texttt{xtest123/testrepo} to yours
    \texttt{organization/repository}
  \end{itemize}
\item
  resetting script \texttt{miqbot\_reset.sh}

  \begin{itemize}
  \tightlist
  \item
    occurrence of \texttt{xtest123/testrepo} to yours
    \texttt{organization/repository}
  \end{itemize}
\item
  configuration file
  \texttt{\textasciitilde{}/miq\_bot/config/settings/development.local.yml}

  \begin{itemize}
  \tightlist
  \item
    github\_credentials → username
  \item
    github\_credentials → password
  \item
    occurrence of \texttt{xtest123/testrepo} to yours
    \texttt{organization/repository}
  \end{itemize}
\end{itemize}

\subsubsection{Setup}\label{setup}

\begin{enumerate}
\def\labelenumi{\arabic{enumi}.}
\tightlist
\item
  Download and install \href{https://www.virtualbox.org/}{Oracle VM
  VirtualBox}
\item
  Download \href{https://www.centos.org/}{CentOS 7}

  \begin{itemize}
  \tightlist
  \item
    \href{http://isoredirect.centos.org/centos/7/isos/x86_64/CentOS-7-x86_64-Minimal-1708.iso}{CentOS-7-x86\_64-Minimal-1708.iso}
  \end{itemize}
\item
  Set up the virtual machine - follow the instructions in
  \href{https://github.com/europ/VUTBR-FIT-BT-IMPL/blob/master/VM_SETUP.md}{VM\_SETUP.md}
\item
  Execute the following commands in the following order from top to
  bottom:

  \begin{itemize}
  \tightlist
  \item
    \texttt{cd\ /root}
  \item
    \texttt{yum\ install\ -y\ git}
  \item
    \texttt{git\ clone\ https://github.com/europ/VUTBR-FIT-BT-IMPL.git}
  \item
    \texttt{cd\ VUTBR-FIT-BT-IMPL}
  \item
    \texttt{./install.sh}
  \item
    \texttt{./rbenv.sh}
  \item
    \texttt{./miqbot\_init.sh}
  \item
    \texttt{cd\ \textasciitilde{}/miq\_bot}
  \item
    Start the bot (\texttt{git\ checkout\ master} - to check that
    everything works correctly):

    \begin{itemize}
    \tightlist
    \item
      Start all Sidekiq workers together

      \begin{itemize}
      \tightlist
      \item
        \texttt{bundle\ exec\ foreman\ start}
      \end{itemize}
    \item
      Start Sidekiq workers separately (open 3 shells)

      \begin{itemize}
      \tightlist
      \item
        \texttt{bundle\ exec\ rails\ s}
      \item
        \texttt{bundle\ exec\ sidekiq\ -q\ miq\_bot}
      \item
        \texttt{bundle\ exec\ sidekiq\ -q\ miq\_bot\_glacial}
      \end{itemize}
    \end{itemize}
  \end{itemize}
\end{enumerate}

\textbf{NOTE}

\begin{itemize}
\tightlist
\item
  If you would like to access the Sidekiq idle (dashboard) from your
  outer browser (not in virtual machine) then you have to launch the
  \texttt{bundle\ exec\ rails\ s} on address \texttt{0.0.0.0} inside of
  the virtual machine and on some port e.g. \texttt{3000}. After this
  you will be able to access the Sidekiq idle via
  \texttt{http://\textless{}VM\_IP\textgreater{}:\textless{}PORT\textgreater{}}
  e.g. \texttt{http://192.168.56.12:3000}.

  \begin{enumerate}
  \def\labelenumi{\arabic{enumi}.}
  \tightlist
  \item
    \texttt{service\ firewalld\ stop}
  \item
    \texttt{bundle\ exec\ rails\ s\ -b\ 0.0.0.0\ -p\ 3000}
  \end{enumerate}
\item
  At bot termination via SIGINT it is possible to get him into unstable
  state. Besides that, you may also face an ``actor crashed'' error too.
  This could be solved by reseting the bot which also require to launch
  him firstly on master branch.

  \begin{enumerate}
  \def\labelenumi{\arabic{enumi}.}
  \tightlist
  \item
    \texttt{\textasciitilde{}/VUTBR-FIT-BT-IMPL/miqbot\_reset.sh}
  \item
    \texttt{cd\ \textasciitilde{}/miq\_bot\ \&\&\ git\ checkout\ master}
  \item
    Launch
  \end{enumerate}
\end{itemize}

\subsubsection{Pull Requests}\label{pull-requests}

To check the pull request functionality you have to
\texttt{git\ checkout} the pertaining branch of the pull request which
can be found in the footnotes of the
\href{https://github.com/europ/VUTBR-FIT-BT/blob/master/PDF/thesis.pdf}{thesis}.

\subsubsection{ManageIQ Bot}\label{manageiq-bot}

The original setup of the
\href{https://github.com/ManageIQ/miq_bot}{ManageIQ bot} can be found
\href{https://github.com/ManageIQ/miq_bot\#development}{here}.

\subsubsection{Bachelor's Thesis}\label{bachelors-thesis}

Bachelor's Thesis is available
\href{https://github.com/europ/VUTBR-FIT-BT}{here}.
