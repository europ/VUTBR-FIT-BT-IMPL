\subsection{Setup Wizard}\label{setup-wizard}

\begin{enumerate}
\def\labelenumi{\arabic{enumi}.}
\tightlist
\item
  Run installed \texttt{Oracle\ VM\ VirtualBox}
\item
  Click on \texttt{New}
\item
  Customize the virtual machine

  \begin{itemize}
  \tightlist
  \item
    add name \texttt{test}
  \item
    choose type \texttt{Linux}
  \item
    choose version \texttt{Other\ Linux\ (64bit)}
  \end{itemize}
\item
  Set memory size e.g. \texttt{4096} MB
\item
  Choose \texttt{Create\ a\ virtual\ hard\ disk\ now} and click on
  \texttt{Create}
\item
  Choose \texttt{VDI\ (VirtualBox\ Disk\ Image)} and press \texttt{Next}
\item
  Choose \texttt{Dynamically\ allocated} and press \texttt{Next}
\item
  Set the size of the virtual hard disk to \texttt{30\ GB} and click on
  \texttt{Create}
\item
  Select the \texttt{test} virtual machine and click on
  \texttt{Settings}
\item
  Choose \texttt{Storage} option in the left side panel
\item
  In \texttt{Storage\ Tree} select the \texttt{Empty} option
\item
  In \texttt{Attributes} click on the button with CD icon
\item
  Select the \texttt{Choose\ Virtual\ Optical\ Disk\ File...} and
  navigate to the downloaded ISO file

  \begin{itemize}
  \tightlist
  \item
    navigate to \texttt{CentOS-7-x86\_64-Minimal-1708.iso}
  \end{itemize}
\item
  Click on \texttt{OK}
\item
  Again select the \texttt{test} virtual machine and click on
  \texttt{Settings}
\item
  Choose \texttt{Network} option in the left side panel
\item
  Select the \texttt{Adapter\ 2} tab
\item
  Check the box option for \texttt{Enable\ Network\ Adapter}
\item
  For \texttt{Attached\ to:} select \texttt{Host-only\ Adapter}
\item
  Click on \texttt{OK}
\item
  Again select the \texttt{test} virtual machine and click on
  \texttt{Start}
\item
  Follow the instruction in
  \href{https://github.com/europ/VUTBR-FIT-BT-IMPL/blob/master/VM_INSTALL.md}{\texttt{Install\ Wizard}}
\end{enumerate}
